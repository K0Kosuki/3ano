%[1] Os alunos dever~ao explicar e interpretar as equac~oes do modelo, referindo:
%a) Qual o signicado de cada uma das constantes a; b; k; c; d; `;
%b) Qual o signicado de cada um dos termos em cada equac~ao (basta fazer para uma das
%equac~oes);
%c) Como evoluiria cada uma destas especies na aus^encia da outra.

%Apresenta¸c˜ao e explica¸c˜ao do modelo. Classifica¸c˜ao das equa¸c˜oes. Significado das equa¸c˜oes e
%dos termos envolvidos nas equa¸c˜oes. O que o modelo %escreve. Aplica¸c˜oes usuais do modelo.
%Eventuais limita¸c˜oes do modelo.
%Apresenta¸c˜ao e explica¸c˜ao do tema. Dedu¸c˜ao de f´ormulas ou de procedimentos (algoritmos) de
%contagem Resultados sobre o tema. Casos particulares, se aplic´avel. Constru¸c˜ao e justifica¸c˜ao
%de fun¸c˜oes geradoras utilizadas. Resultados auxiliares utilizados, etc etc


\section{Evolução de uma espécie na ausência da outra (espécie isolada)}
\subsection{Equações do modelo}
\begin{equation}
\left\{
\begin{array}{l}
P'(t)=[a-bP(t)]\;P(t)]=aP(t)\left[1-\frac{P(t)}{s}\right]  \medskip  \\
Q'(t)=[c-dQ(t)]\;Q(t)]=cQ(t)\left[1-\frac{Q(t)}{r}\right]
\end{array}
\right.
\end{equation}

%\bigskip
%\begin{equation}
%\begin{array}{l}
%P(0)>s \rightarrow P'(0)=aP(0)\left[1-\frac{P(0)}{s}\right]=-C \cdot aP(0) \medskip   \\
%P(0)<s \rightarrow P'(0)=aP(0)\left[1-\frac{P(0)}{s}\right]=C \cdot aP(0)  \medskip  \\
%P(0)=s \rightarrow P'(0)=aP(0)\left[1-\frac{P(0)}{s}\right]=0 \cdot aP(0)   \\
%\end{array}
%\end{equation}


\subsection{Análise do modelo}
\noindent
Após analisar matematicamente a equação tal retiramos as seguintes conclusões:\smallskip\newline
Quando  $P(0)>s$, a população diminui.\newline
Quando  $P(0)<s$, a população aumenta.\newline
Quando  $P(0)=s$, a população estabiliza.




\noindent
\section{Evolução de uma espécie em competição com a outra}
\subsection{Equações do modelo}
%[1] Os alunos dever~ao explicar e interpretar as equac~oes do modelo, referindo:
%a) Qual o signicado de cada uma das constantes a; b; k; c; d; `;
%b) Qual o signicado de cada um dos termos em cada equac~ao (basta fazer para uma das
%equac~oes);
%c) Como evoluiria cada uma destas especies na aus^encia da outra.

%Apresenta¸c˜ao e explica¸c˜ao do modelo. Classifica¸c˜ao das equa¸c˜oes. Significado das equa¸c˜oes e
%dos termos envolvidos nas equa¸c˜oes. O que o modelo %escreve. Aplica¸c˜oes usuais do modelo.
%Eventuais limita¸c˜oes do modelo.
%Apresenta¸c˜ao e explica¸c˜ao do tema. Dedu¸c˜ao de f´ormulas ou de procedimentos (algoritmos) de
%contagem Resultados sobre o tema. Casos particulares, se aplic´avel. Constru¸c˜ao e justifica¸c˜ao
%de fun¸c˜oes geradoras utilizadas. Resultados auxiliares utilizados, etc etc
\noindent
A evolução das populações é descrita pelo modelo
\begin{equation}
\left\{
\begin{array}{l}
P'(t)=[a-bP(t)-kQ(t)]\;P(t)]  \medskip  \\
Q'(t)=[c-dQ(t)-\ell P(t)]\;Q(t)]
\end{array}
\right.
\end{equation}

%\medskip
\noindent
onde a,b,k,c,d,$\ell$ são constantes positivas.

\subsection{Análise do modelo}
%Qual o signicado de cada uma das constantes a; b; k; c; d; `;
\noindent
As constantes $a$ e $c$ correspondem às taxas de crescimento intrínseco das espécies.\newline
As constantes $b$ e $d$ correspondem às taxas inibidoras de crescimento das espécies.\newline
As constantes $k$ e $\ell$ correspondem ao efeito competitivo de uma espécie sobre a outra.
\medskip
%Qual o signicado de cada um dos termos em cada equac~ao (basta fazer para uma das« equac~oes);




%c) Como evoluiria cada uma destas especies na aus^encia da outra.
%Link para Equações https://www.ine.pt/revstat/pdf/Paper1_BrilhanteETAL.pdf
%https://www.khanacademy.org/science/ap-biology/ecology-ap/population-ecology-ap/a/exponential-logistic-growth IMPORTANTE!






%\noindent
%O crescimento exponencial ocorre quando existe poucos indivíduos e muitos recursos. Mas quando o número de indivíduos aumenta o suficiente, os recursos começam a ficar escassos, diminuindo assim a taxa de crescimento. Eventualmente, a taxa de crescimento irá estabilizar.\newline
%O tamanho da população no qual o crescimento se estabiliza representa o tamanho máximo da população que um determinado ambiente pode suportar, que é denominado de nível de saturação que corresponde ao número máximo de indivíduos.
\medskip\noindent Após analisarmos o sistema verificamos que existes três possibilidades:
\begin{enumerate}[ {(}1{)} ]
\item Ocorre a extinção de ambas as espécies.
\item Uma espécie sobrevive, enquanto a outra se extingue.
\item Ambas as espécies sobrevivem, e encontram uma “convivência estável”.
\end{enumerate}


