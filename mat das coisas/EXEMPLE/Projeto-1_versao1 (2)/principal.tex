%Classe do documento
\documentclass[11pt,a4paper]{article}
\usepackage[dvipdf]{epsfig}
\usepackage{amsfonts,amsmath,amssymb,amsthm}

\usepackage[pdftex,
            pdfauthor={Your Name},
            pdftitle={The Title},
            pdfsubject={The Subject},
            pdfkeywords={Some Keywords},
            pdfproducer={Latex with hyperref, or other system},
            pdfcreator={pdflatex, or other tool}]{hyperref}
% Portuges
\usepackage[portuges]{babel}
% Cores
\usepackage{color}
\usepackage{enumerate}
\usepackage{hyperref}
% Carateres internacionais.
\usepackage[utf8]{inputenc}

\usepackage[gen]{eurosym}
\usepackage{enumerate}

% Fontes internacionais de melhor qualidade.
\usepackage[T1]{fontenc}

% Incluir imagens.
\usepackage{graphicx}

% Tabelas
\usepackage{tabularx}

%Tabelas multi-coluna
\usepackage{multirow}

% When it comes to bibliography management in LATEX, the package natbib is a package for customising citations (especially author-year citation schemes) when using BibTeX. 
%\usepackage{natbib}
\usepackage{biblatex}

% Caminho para a bibliografia
\addbibresource{bibliografia.bib}

% The package enables the user to typeset programs (programming code).
\usepackage{listings}

% The caption package provides many ways to customise the captions in floating environments like figure and table, and cooperates with many other packages.
\usepackage{subcaption}

%The package enables the user to produce and typeset one or more indexes simultaneously with a document.
\usepackage{imakeidx}

%Usado para pagestyles
\usepackage{fancyhdr}

%Tipo e margens do documento
%\usepackage[a4paper,top=3cm,bottom=3cm,left=3cm,right=3cm]{geometry}

% Para definir a mancha de texto e a sua posicao
\hoffset -0.7in
\voffset -0.5in
\addtolength{\textwidth}{3.5cm}
\addtolength{\textheight}{2cm}

% Define figures como o path para as imagens.
\graphicspath{ {imagens/} }

% A general purpose hierarchical index generator; it accepts one or more input files.
\makeindex




% Para os Cabeçalhos            
\pagestyle{fancy}
\fancyhf{}
\fancyhead[R]{\leftmark}
\fancyhead[L]{\rightmark}
\fancyfoot[R]{\thepage}
\lhead{CSE 101}
\rhead{\leftmark}
\lhead{\rightmark}
\fancyfoot[R]{\thepage}
%------------------------------



%%%%%%%%%%%%%
% Para definir a mancha de texto e a sua posicao
\hoffset -0.7in
\voffset -0.5in
\addtolength{\textwidth}{3.5cm}
\addtolength{\textheight}{2cm}

%%%%%%%%%%%%%%%%%%%
% Simbolos
\def\RR{\mathbb R}
\def\QQ{\mathbb Q}
\def\NN{\mathbb N}
\def\ZZ{\mathbb Z}
\def\PP{\mathbb P}
\def\CC{\mathbb C}

\def\u{\mbox{\boldmath$u$}}
\def\v{\mbox{\boldmath$v$}}
\def\q{\mbox{\boldmath$q$}}

%%%%%%%%%%%%%%%%%%%
% Para definir cores

\definecolor{cinz}{cmyk}{0,0,0,.5}
\definecolor{cinzz}{cmyk}{0,0,0,.75}
\definecolor{violeta}{cmyk}{0.3,1,0,0}
\definecolor{cast}{cmyk}{0,.4,1,.4}
\definecolor{ggreen}{cmyk}{1,     0,      1,      0}
\definecolor{ror}{cmyk}{0, 0.77, 0.87, 0}
\definecolor{amber(sae/ece)}{rgb}{1.0, 0.49, 0.0}
\definecolor{navyblue}{rgb}{0.2, 0.2, 0.6}
\definecolor{azul}{rgb}{0.2, 0.29996, 0.8 }
\definecolor{myred}{cmyk}{0.1, 1, 0.5, 0}

%%%%%%%%%%%%%%%%%%%
% Teoremas e outros ambiente

\theoremstyle{plain}
\newtheorem{thm}{Theorem}[section]
\newtheorem{lem}{Lemma}[section]
\newtheorem{prop}[thm]{Proposition}
\newtheorem*{cor}{Corollary}

\theoremstyle{definition}
\newtheorem{defn}{Definition}[section]
\newtheorem{conj}{Conjecture}[section]
\newtheorem{exmp}{Example}[section]

\theoremstyle{remark}
\newtheorem*{rem}{Remark}
\newtheorem*{note}{Note}

%%%%%%%%%%%%%%%%%%%

\newcommand{\dps}{\displaystyle}
\allowdisplaybreaks[1]

%%%%%%%%%%%%%%%%%%%
% Operadores

\DeclareMathOperator{\sen}{sen}
%\DeclareMathOperator{\cos}{cos}
\DeclareMathOperator{\tg}{tg}
\DeclareMathOperator{\cotg}{cotg}

\DeclareMathOperator{\asen}{arcsen}
\DeclareMathOperator{\arcos}{arccos}
\DeclareMathOperator{\arctg}{arctg}
\DeclareMathOperator{\arccotg}{arccotg}

\DeclareMathOperator{\sh}{sh}
\DeclareMathOperator{\ch}{ch}
\DeclareMathOperator{\tah}{th}
%\DeclareMathOperator{\coth}{coth}

\DeclareMathOperator{\agc}{argch}
\DeclareMathOperator{\ags}{argsh}
\DeclareMathOperator{\agt}{argth}
\DeclareMathOperator{\agct}{argcoth}

%%%%%%%%%%%%%%

% Inicio do documento.
\begin{document}

%NAO SEI BEM PARA QUE SERVE
%\baselineskip=1.4em

% Capa
%
%Dong Xuyong A92960 Licenciatura em Gestão de Sistemas de Informação
%Inês José A88690 Licenciatura em Economia
%Leandro Pereira, Ciências Politicas
%Luís Zhou, A88608,Licenciatura em Estatística Aplicada.
%Ricardo Oliveira A73055 Mestrado Integrado em Engenharia Eletrónica Industrial e Computadores
%GRUPO 3
%Matemática das Coisas
%Departamento de Matemática, Universidade do Minho
%Ana Jacinta Soares
%Projecto 1B - Dinâmica de populações em competição


\begin{titlepage}
\begin{figure}[!htb]
    \centering
    \includegraphics[keepaspectratio=true,scale=0.75]{umfinal.jpg}
\end{figure}

\begin{center}
    \LARGE{Universidade do Minho\\Departamento de Matemática\\Matemática das Coisas}
\end{center}

\vspace{10mm}
\begin{center}
\noindent\rule{13cm}{0.4pt} \linebreak
    {\LARGE{\bf Projeto 1,\\\vspace{5mm}Dinâmica de populações em competição}} \linebreak  
\noindent\rule{13cm}{0.4pt} \linebreak
\end{center}
\vspace{10mm}

\begin{center}
%	{\large{\bf{Grupo 3:}}{\normalsize\vspace{3mm}
%	\\ \large{Dong Xuyong a92960, Licenciatura em Gestão de Sistemas de Informação\vspace{3mm}}}}
%	\\ \large{Inês José a88690, Licenciatura em Economia\vspace{3mm}}}}
	%\\ \large{Leandro Pereira a73055, Mestrado Integrado em Engenharia Eletrónica Industrial e Computadores\vspace{3mm}}}}
	%\\ \large{Luís Zhou a88608, Licenciatura em Estatística Aplicada\vspace{3mm}}}}
%	\\ \large{Ricardo Oliveira a73055, Mestrado Integrado em Engenharia Eletrónica Industrial e Computadores\vspace{3mm}}}}
		{\large{\bf{Grupo 3:}}{\normalsize\vspace{3mm}
	\\ \large{Dong Xuyong a92960 L.G.S.I\\ Inês José a88690, L.E \\ Leandro Pereira a73055 L.C.P \\ Luís Zhou a88608 L.E.A \\ Ricardo Oliveira a73055 M.I.E.E.I.C \vspace{3mm}}}}

\end{center}
\hfill
\begin{center}
	{\large{\bf{Professora:}}{\normalsize\vspace{3mm} \\ \large{Ana Jacinta Soares\\ }}}
\end{center}

\vspace{20mm}
\hrulefill
\\\centering{\large{31 de Março de 2022}}

\end{titlepage}

% Indice
%\tableofcontents
%\pagebreak

% Indice de figuras
%\listoffigures
%\pagebreak

%Indice de Tabelas
%\listoftables
%\pagebreak

% Teoria
%%Base teorica para o projeto (OPCIONAL)

% Modelo ou Tema
%%[1] Os alunos dever~ao explicar e interpretar as equac~oes do modelo, referindo:
%a) Qual o signicado de cada uma das constantes a; b; k; c; d; `;
%b) Qual o signicado de cada um dos termos em cada equac~ao (basta fazer para uma das
%equac~oes);
%c) Como evoluiria cada uma destas especies na aus^encia da outra.

%Apresenta¸c˜ao e explica¸c˜ao do modelo. Classifica¸c˜ao das equa¸c˜oes. Significado das equa¸c˜oes e
%dos termos envolvidos nas equa¸c˜oes. O que o modelo %escreve. Aplica¸c˜oes usuais do modelo.
%Eventuais limita¸c˜oes do modelo.
%Apresenta¸c˜ao e explica¸c˜ao do tema. Dedu¸c˜ao de f´ormulas ou de procedimentos (algoritmos) de
%contagem Resultados sobre o tema. Casos particulares, se aplic´avel. Constru¸c˜ao e justifica¸c˜ao
%de fun¸c˜oes geradoras utilizadas. Resultados auxiliares utilizados, etc etc


\section{Evolução de uma espécie na ausência da outra (espécie isolada)}
\subsection{Equações do modelo}
\begin{equation}
\left\{
\begin{array}{l}
P'(t)=[a-bP(t)]\;P(t)]=aP(t)\left[1-\frac{P(t)}{s}\right]  \medskip  \\
Q'(t)=[c-dQ(t)]\;Q(t)]=cQ(t)\left[1-\frac{Q(t)}{r}\right]
\end{array}
\right.
\end{equation}

%\bigskip
%\begin{equation}
%\begin{array}{l}
%P(0)>s \rightarrow P'(0)=aP(0)\left[1-\frac{P(0)}{s}\right]=-C \cdot aP(0) \medskip   \\
%P(0)<s \rightarrow P'(0)=aP(0)\left[1-\frac{P(0)}{s}\right]=C \cdot aP(0)  \medskip  \\
%P(0)=s \rightarrow P'(0)=aP(0)\left[1-\frac{P(0)}{s}\right]=0 \cdot aP(0)   \\
%\end{array}
%\end{equation}


\subsection{Análise do modelo}
\noindent
Após analisar matematicamente a equação tal retiramos as seguintes conclusões:\smallskip\newline
Quando  $P(0)>s$, a população diminui.\newline
Quando  $P(0)<s$, a população aumenta.\newline
Quando  $P(0)=s$, a população estabiliza.




\noindent
\section{Evolução de uma espécie em competição com a outra}
\subsection{Equações do modelo}
%[1] Os alunos dever~ao explicar e interpretar as equac~oes do modelo, referindo:
%a) Qual o signicado de cada uma das constantes a; b; k; c; d; `;
%b) Qual o signicado de cada um dos termos em cada equac~ao (basta fazer para uma das
%equac~oes);
%c) Como evoluiria cada uma destas especies na aus^encia da outra.

%Apresenta¸c˜ao e explica¸c˜ao do modelo. Classifica¸c˜ao das equa¸c˜oes. Significado das equa¸c˜oes e
%dos termos envolvidos nas equa¸c˜oes. O que o modelo %escreve. Aplica¸c˜oes usuais do modelo.
%Eventuais limita¸c˜oes do modelo.
%Apresenta¸c˜ao e explica¸c˜ao do tema. Dedu¸c˜ao de f´ormulas ou de procedimentos (algoritmos) de
%contagem Resultados sobre o tema. Casos particulares, se aplic´avel. Constru¸c˜ao e justifica¸c˜ao
%de fun¸c˜oes geradoras utilizadas. Resultados auxiliares utilizados, etc etc
\noindent
A evolução das populações é descrita pelo modelo
\begin{equation}
\left\{
\begin{array}{l}
P'(t)=[a-bP(t)-kQ(t)]\;P(t)]  \medskip  \\
Q'(t)=[c-dQ(t)-\ell P(t)]\;Q(t)]
\end{array}
\right.
\end{equation}

%\medskip
\noindent
onde a,b,k,c,d,$\ell$ são constantes positivas.

\subsection{Análise do modelo}
%Qual o signicado de cada uma das constantes a; b; k; c; d; `;
\noindent
As constantes $a$ e $c$ correspondem às taxas de crescimento intrínseco das espécies.\newline
As constantes $b$ e $d$ correspondem às taxas inibidoras de crescimento das espécies.\newline
As constantes $k$ e $\ell$ correspondem ao efeito competitivo de uma espécie sobre a outra.
\medskip
%Qual o signicado de cada um dos termos em cada equac~ao (basta fazer para uma das« equac~oes);




%c) Como evoluiria cada uma destas especies na aus^encia da outra.
%Link para Equações https://www.ine.pt/revstat/pdf/Paper1_BrilhanteETAL.pdf
%https://www.khanacademy.org/science/ap-biology/ecology-ap/population-ecology-ap/a/exponential-logistic-growth IMPORTANTE!






%\noindent
%O crescimento exponencial ocorre quando existe poucos indivíduos e muitos recursos. Mas quando o número de indivíduos aumenta o suficiente, os recursos começam a ficar escassos, diminuindo assim a taxa de crescimento. Eventualmente, a taxa de crescimento irá estabilizar.\newline
%O tamanho da população no qual o crescimento se estabiliza representa o tamanho máximo da população que um determinado ambiente pode suportar, que é denominado de nível de saturação que corresponde ao número máximo de indivíduos.
\medskip\noindent Após analisarmos o sistema verificamos que existes três possibilidades:
\begin{enumerate}[ {(}1{)} ]
\item Ocorre a extinção de ambas as espécies.
\item Uma espécie sobrevive, enquanto a outra se extingue.
\item Ambas as espécies sobrevivem, e encontram uma “convivência estável”.
\end{enumerate}




% Estudo e exploraçao do modelo ou do tema de combinatoria
%%Pretende-se, ainda, que os alunos determinem os pontos de equilbrio, em geral, e estudem a
%sua estabilidade.

%Resolu¸c˜ao anal´ıtica das equa¸c˜oes, quando aplic´avel. Representa¸c˜ao das solu¸c˜oes. Campo de direc¸c˜oes. Propriedades e significado das equa¸c˜oes ou das solu¸c˜oes. Pontos de equil´ıbrio. Diagrama
%de fases. An´alise da estabilidade dos pontos de equil´ıbrio. Outros estudos pertinentes para o
%modelo escolhido. Pode fazer sentido estudar o modelo em diversas fases: por exemplo incluindo
%ou n˜ao diversos efeitos nas equa¸c˜oes.
%Para um tema de combinat´oria, poder´a fazer sentido apenas exemplos, exerc´ıcios, problemas de
%aplica¸c˜ao, etc

\subsection{Determinação dos pontos de equilíbrio}
Recorrendo a técnicas da \textbf{Teoria dos Sistemas Dinâmicos}, podemos prever estas situações, fazendo uma \textbf{análise quantitativa da solução do modelo}, calculando os \textbf{pontos de equilíbrio} e a sua \textbf{estabilidade}.\newline
Estudo do \textbf{sistema dinâmico “reduzido”}:

\begin{equation}
\left\{
\begin{array}{l}
P'=(a-bP-kQ)P  \medskip  \\
Q'=(c-dQ-\ell P)Q
\end{array}
\right.
\label{eq:reduzida}
\end{equation}
\bigskip\medskip\newline\noindent Procurando os \textbf{pontos de equilíbrio}: \bigskip\medskip\newline\noindent
$
\left\{
\begin{array}{l}
(a-bP-kQ)P=0  \medskip  \\
(c-dQ-\ell P)Q=0
\end{array}
\right.
\Leftrightarrow\;\;
\left\{
\begin{array}{l}
P=0 \vee a-bP-kQ=0  \medskip  \\
Q=0 \vee c-dQ-\ell P=0
\end{array}
\right.
$
\bigskip\medskip\newline\noindent Obtemos os \textbf{sistemas de equações}, sendo que posteriormente, determinamos as suas \textbf{soluções}: \bigskip\medskip\newline\noindent
$
\left\{
\begin{array}{l}
P=0  \medskip  \\
Q=0
\end{array}
\right.
\vee\;\;
\left\{
\begin{array}{l}
P=0  \medskip  \\
Q=c-dQ-\ell P=0
\end{array}
\right.
\vee\;\;
\left\{
\begin{array}{l}
a-bP-kQ=0  \medskip  \\
Q=0
\end{array}
\right.
\vee\;\;
\left\{
\begin{array}{l}
a-bP-kQ=0  \medskip  \\
c-dQ-\ell P=0
\end{array}
\right.
$
\bigskip\medskip\newline\noindent Cálculos necessários para determinação das soluções do \textbf{primeiro} sistema de equações:  \bigskip\medskip\newline\noindent
$
\left\{
\begin{array}{l}
P=0  \medskip  \\
Q=0
\end{array}
\right.
$
\bigskip\medskip\newline\noindent Cálculos necessários para determinação das soluções do \textbf{segundo} sistema de equações:  \bigskip\medskip\newline\noindent
$
\left\{
\begin{array}{l}
P=0  \medskip  \\
c-dQ-\ell P=0
\end{array}
\right.
\Leftrightarrow\;\;
\left\{
\begin{array}{l}
P=0  \medskip  \\
Q=\frac{-c}{-d}
\end{array}
\right.
\Leftrightarrow\;\;
\left\{
\begin{array}{l}
P=0  \medskip  \\
Q=\frac{c}{d}
\end{array}
\right.
$
\bigskip\medskip\newline\noindent Cálculos necessários para determinação das soluções do \textbf{terceiro} sistema de equações: \bigskip\medskip\newline\noindent
$
\left\{
\begin{array}{l}
a-bP-kQ=0  \medskip  \\
Q=0
\end{array}
\right.
\Leftrightarrow\;\;
\left\{
\begin{array}{l}
P=\frac{-a}{-b}  \medskip  \\
Q=0
\end{array}
\right.
\Leftrightarrow\;\;
\left\{
\begin{array}{l}
P=\frac{a}{b}  \medskip  \\
Q=0
\end{array}
\right.
$
\bigskip\medskip\newline\noindent Cálculos necessários para determinação das soluções do \textbf{quarto} sistema de equações: \bigskip\medskip\newline\noindent
$
\left\{
\begin{array}{l}
a-bP-kQ=0  \medskip  \\
Q=c-dQ-\ell P
\end{array}
\right.
\Leftrightarrow\;\;
\left\{
\begin{array}{l}
P=\frac{-a+kQ}{-b}  \medskip  \\
Q=c-dQ-\ell P
\end{array}
\right.
\Leftrightarrow\;\;
\left\{
\begin{array}{l}
P=\frac{a-kQ}{b}  \medskip  \\
Q=c-dQ-\ell P
\end{array}
\right.
\Leftrightarrow\;\;
$
\bigskip\newline\noindent
$
\left\{
\begin{array}{l}
P=\frac{a-kQ}{b}  \medskip  \\
c-dQ-\ell \left(\frac{a-kQ}{b}\right)=0
\end{array}
\right.
\Leftrightarrow\;\;
\left\{
\begin{array}{l}
P=\frac{a-kQ}{b}  \medskip  \\
c-dQ-\frac{\ell a}{b}+\frac{\ell kQ}{b}=0
\end{array}
\right.
\Leftrightarrow\;\;
\left\{
\begin{array}{l}
P=\frac{a-kQ}{b}  \medskip  \\
Q\left(-d+\frac{\ell k}{b}\right)=\frac{\ell a}{b}-c
\end{array}
\right.
\Leftrightarrow\;\;
$
%TRE
\bigskip\newline\noindent
$
\left\{
\begin{array}{l}
P=\frac{a-kQ}{b}  \medskip  \\
Q=\frac{\frac{\ell a}{b}-c}{\frac{\ell k}{b}-d}
\end{array}
\right.
=\;\;
\left\{
\begin{array}{l}
P=\frac{a-kQ}{b}  \medskip  \\
Q=\frac{\frac{\ell a}{b}-cb}{\frac{\ell k}{b}-db}
\end{array}
\right.
\Leftrightarrow\;\;
\left\{
\begin{array}{l}
P=\frac{a-kQ}{b}  \medskip  \\
Q=\frac{\ell a-cb}{\ell k-db}
\end{array}
\right.
\Leftrightarrow\;\;
\left\{
\begin{array}{l}
P=\frac{a-kQ}{b}  \medskip  \\
Q=\frac{a \ell-bc}{k \ell-bd}
\end{array}
\right.
\Leftrightarrow\;\;
$
%LASTTTT
\bigskip\newline\noindent
$
\left\{
\begin{array}{l}
P=\frac{a-k\left(\frac{al-bc}{kl-bd}\right)}{b}  \medskip  \\
Q=\frac{a \ell-bc}{k \ell-bd}
\end{array}
\right.
\Leftrightarrow\;\;
\left\{
\begin{array}{l}
P=\frac{\left(\frac{ak \ell - abd -ka \ell + kbc}{k \ell-bd}\right)}{b}  \medskip  \\
Q=\frac{a \ell-bc}{k \ell-bd}
\end{array}
\right.
\Leftrightarrow\;\;
\left\{
\begin{array}{l}
P=\frac{\left(\frac{ak \ell -ak \ell -abd +bck}{k \ell-bd}\right)}{b}  \medskip  \\
Q=\frac{a \ell-bc}{k \ell-bd}
\end{array}
\right.
\Leftrightarrow\;\;
$
\bigskip\newline\noindent
$
\left\{
\begin{array}{l}
P=\frac{\left(\frac{-abd +bck}{k \ell-bd}\right)}{b}  \medskip  \\
Q=\frac{a \ell-bc}{k \ell-bd}
\end{array}
\right.
\Leftrightarrow\;\;
\left\{
\begin{array}{l}
P=\frac {b(-abd+bck)}{kl-bd}  \medskip  \\
Q=\frac{a \ell-bc}{k \ell-bd}
\end{array}
\right.
\Leftrightarrow\;\;
\left\{
\begin{array}{l}
P=\frac {-ad+ck}{kl-bd}  \medskip  \\
Q=\frac{a \ell-bc}{k \ell-bd}
\end{array}
\right.
$
%\left(\right)
\bigskip\medskip\newline\noindent Após termos determinado as \textbf{soluções}, obtemos os \textbf{pontos de equilíbrio}:\medskip\newline\noindent
$(P_{1}^{*},Q_{1}^{*})=\left(0,0\right)$\medskip\newline
$(P_{2}^{*},Q_{2}^{*})=\left(0,\frac{c}{d}\right)$\medskip\newline
$(P_{3}^{*},Q_{3}^{*})=\left(\frac{a}{b},0\right)$\medskip\newline
$(P_{4}^{*},Q_{4}^{*})=\left(\frac {-ad+ck}{kl-bd},\frac{a \ell-bc}{k \ell-bd}\right)$

\newpage

\subsection{Estabilidade}
Partimos da equação (\ref{eq:reduzida}) e definimos as funções:\bigskip\medskip\newline\noindent
$
\left\{
\begin{array}{l}
F(P,Q)=(a-bP-kQ)P=aP-bP^2-kQP  \medskip  \\
G(P,Q)=(c-dQ-\ell P)Q=cQ-dQ^2-\ell PQ
\end{array}
\right.
$
\bigskip\medskip\newline\noindent Construímos a \textbf{matriz Jacobiana}:\bigskip\medskip\newline\noindent
$J=
\renewcommand{\arraystretch}{1.25}
\begin{bmatrix}
  \frac{\partial F}{\partial P} & \frac{\partial F}{\partial Q}\medskip  \\
  \frac{\partial G}{\partial P} & \frac{\partial G}{\partial Q}
\end{bmatrix}
=
\renewcommand{\arraystretch}{1.25}
\begin{bmatrix}
  a-2bP-kQ & -kP\medskip  \\
  \ell Q & c-2dQ- \ell P
 \end{bmatrix}
$
\bigskip\medskip\newline\noindent Calculamos as respetivas matrizes Jacobianas em cada um dos pontos de equilíbrio, para assim, podermos estudar a estabilidade nesses pontos. \bigskip\medskip\newline\noindent
J$(P_{1}^{*},Q_{1}^{*})=\left(0,0\right)$ =$
\renewcommand{\arraystretch}{1.25}
\begin{bmatrix}
  a & 0\medskip  \\
  0 & c
\end{bmatrix}
$
\medskip\bigskip \newline
J$(P_{2}^{*},Q_{2}^{*})=\left(0,\frac{c}{d}\right)$ =$
\renewcommand{\arraystretch}{1.25}
\begin{bmatrix}
  a-\frac{ck}{d} & 0\medskip  \\
  -\frac{cl}{d} & -c
\end{bmatrix}
$
\medskip\bigskip \newline
J$(P_{3}^{*},Q_{3}^{*})=\left(\frac{a}{b},0\right)$ =$
\renewcommand{\arraystretch}{1.25}
\begin{bmatrix}
  -a & -\frac{ak}{b} \medskip  \\
  0 & c- \frac{a \ell}{b}
\end{bmatrix}
$
\medskip\bigskip \newline
J$(P_{4}^{*},Q_{4}^{*})=\left(\frac {-ad+ck}{kl-bd},\frac{a \ell-bc}{k \ell-bd}\right)$ =$
\renewcommand{\arraystretch}{1.25}
\begin{bmatrix}
  \frac{abd-bck}{k \ell-bd} & \frac{adk-ck^2}{k \ell-bd}\medskip  \\
  \frac{bc \ell- a \ell^2}{k \ell-bd} & \frac{bcd-ad \ell}{k \ell-bd}
\end{bmatrix}
$





\bigskip\medskip\noindent Calculamos os \textbf{valores próprios} das matrizes:\bigskip\newline\noindent
Pela definição do vetor característico $v$ correspondente ao valor característico $\lambda$ temos:\newline
$Av=\lambda v$\newline
Neste caso:\newline
$Av-\lambda v=(A- \lambda I) \cdot v=0$\newline
A equação têm uma solução não nula se, e só se,\newline det$(A- \lambda I)=0$
\newpage
\noindent Cálculo do valor próprio da \textbf{primeira} matriz:\bigskip\medskip\newline\noindent
det$(A- \lambda I)=\renewcommand{\arraystretch}{1.25}
\begin{vmatrix}
  a-\lambda & -c\medskip  \\
  0 & c- \lambda
\end{vmatrix}=\lambda^2-(a+c)\cdot \lambda +ac=(\lambda-c) \cdot (\lambda -a)=0)$\medskip\newline
$\lambda_1=c$ e $\lambda_2=a$
%c
%a
\bigskip\newline\noindent Cálculo do valor próprio da \textbf{segunda} matriz: \bigskip\medskip\newline\noindent
det$(A- \lambda I)=\renewcommand{\arraystretch}{1.25}
\begin{vmatrix}
  \frac{ad-ck}{d}-\lambda & 0\medskip  \\
  \frac{-cl}{d} & {-c-\lambda}
\end{vmatrix}=\lambda^2-\frac{ad-cd-ck}{d}\cdot \lambda  -\frac{acd-c^2k}{d}=$\bigskip\newline
$\frac{1}{d} \cdot (d\lambda^2 -(ad-cd-ck) \cdot \lambda - (acd-c^2k))=d \cdot \frac{1}{d} \cdot (\lambda +c ) \cdot \left(\lambda- \frac{ad-ck}{d}\right)=0$\medskip\newline
$\lambda_1=-c$ e $\lambda_2=\frac{ad-ck}{d}$
%c
%a
\bigskip\newline\noindent Cálculo do valor próprio da \textbf{terceira} matriz: \bigskip\medskip\newline\noindent
det$(A- \lambda I)=\renewcommand{\arraystretch}{1.25}
\begin{vmatrix}
  -a-\lambda- & \frac{-ak}{b}\medskip  \\
   0 & \frac{bc-a\ell}{b}-\lambda
\end{vmatrix}=\lambda^2-\frac{ab-bc+a\ell}{b}\cdot \lambda  -\frac{abc-a^2\ell}{b}=$\bigskip\newline
$\frac{1}{b} \cdot (b\lambda^2 +(ab-bc+al) \cdot \lambda - (abc-a^2\ell))=b \cdot \frac{1}{b} \cdot (\lambda +a ) \cdot \left(\lambda- \frac{bc-al}{b}\right)=0$\medskip\newline
$\lambda_1=-a$ e $\lambda_2=\frac{bc-al}{b}$
%c
%a
\bigskip\medskip\newline\noindent Cálculo do valor próprio da \textbf{quarta} matriz: \bigskip\medskip\newline\noindent
det$(A- \lambda I)=\renewcommand{\arraystretch}{1.25}
\begin{vmatrix}
  \frac{-abd+bck}{bd-k\ell}-\lambda & \frac{ck^2-adk}{bd-k\ell}\medskip  \\
  \frac{a\ell^2-bcl}{bd-kl} & \frac{-bcd+ad\ell}{bd-k\ell} -\lambda
\end{vmatrix}$\medskip\newline
Não há soluções racionais.\bigskip\newline
Depois de termos calculado os valores próprios destas matrizes, obtemos:\medskip\newline
Para J$(P_{1}^{*},Q_{1}^{*})=\left(0,0\right)$, são $\lambda_(1)_(1)=c$ e $\lambda_(1)_(2)=a$.\medskip\newline
Para J$(P_{2}^{*},Q_{2}^{*})=\left(0,\frac{c}{d}\right)$, são $\lambda_(2)_(1)=-c$ e $\lambda_(2)_(2)=\frac{ad-ck}{d}$.\medskip\newline
Para J$(P_{3}^{*},Q_{3}^{*})=\left(\frac{a}{b},0\right)$, são $\lambda_(3)_(1)=-a$ e $\lambda_(3)_(2)=\frac{bc-al}{b}$.\medskip\newline
Para J$(P_{4}^{*},Q_{4}^{*})=\left(\frac {-ad+ck}{kl-bd},\frac{a \ell-bc}{k \ell-bd}\right)$, não há soluções racionais.\medskip\newline


\noindent Nada se pode concluir sobre a estabilidade dos pontos de equilíbrio, pois para isso, seria necessário uma análise detalhada usando os vetores próprios.


% Aplicacoes e simulacoes numericas
%\input{Condicoes}

Ricardo Filipe Sousa Oliveira Ricardo Filipe Sousa Oliveira Ricardo Filipe Sousa Oliveira Ricardo Filipe Sousa Oliveira Ricardo Filipe Sousa Oliveira Ricardo Filipe Sousa Oliveira Ricardo Filipe Sousa Oliveira Ricardo Filipe Sousa Oliveira Ricardo Filipe Sousa Oliveira Ricardo Filipe Sousa Oliveira Ricardo Filipe Sousa Oliveira Ricardo Filipe Sousa Oliveira Ricardo Filipe Sousa Oliveira Ricardo Filipe Sousa Oliveira Ricardo Filipe Sousa Oliveira Ricardo Filipe Sousa Oliveira 
%\pagebreak
%\printbibliography
%----------------------------------------------------------------------------------------

\end{document}